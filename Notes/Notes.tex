\documentclass{article}

\usepackage{amsmath}
\usepackage{chemfig}


\title{Notes While Building Computational Context Model}
\author{Aleczander Taylor}
\date{\today}

\begin{document}
\maketitle
\begin{abstract}
  This is an attempt to port Enoch Yeung's Compositional Context Scripts to Julia while 
  expanding them to include the option of adding addition genes, the effect on energetic 
  regions, and to study the effect targeted DNA binding (i.e. Cas effectors) proteins
  have on the results they observed. This is mainly a notebook so that I can keep track of
  the math.
\end{abstract}

\section{Introduction}
  We sought to recreate the work of Yeung et al.\ in regard to quantification of
  supercoiliing based compositional context effects in regard to synthetic gene networks,
  to then modify it to include several scenarios. These include the introduction of
  targeted DNA binding proteins, such as the Cas family, as well as the effect of backbone
  expression cassettes such as the origin of replication, etc\dots

  Mainly this will focus on my process in developing these ideas.
  
\section{An Aside Regarding Michaelis-Menten Kinetics and Hill Functions.}
\subsection{Victory Henri}
  Henri is generally thought to be the first person to create a mathematical framework to 
  explain enzymatic activity. A Russian physical chemist born in France due to an
  illegitimate conception, of whom latter returned to Russia (specifically Saint 
  Petersburg). He was the relative of notable control theorist Aleksander Mikhailovich Lyapanov, 
  composer Sergei Mikhailovich Lyapanov and Slavic liquist Boris Mikhailovich Lyapanov.

  He was eventually be educated in mathematics and Natural Sciences, and earned two PhDs. The first was in 
  psychology and the second in physical Chemistry.

  As a prominent subject of study at the tine, \textbf{Invertase} became the main focus 
  of Henri's Studies. The reaction of which is included below.
  \begin{figure}
    \tiny \chemfig{*6((=CH-OH)- (-OH)- (-OH)- (-0H)- (-OH)-O-)}
  \end{figure}
  Which eventually, with the help of Adrian John Brown, leading him to formulate the equation:
  \[
  \dot{x} = \frac{K*\Phi*(a-x)}{1+m*(a-x)+n*x}  
  \]
  Where $x \equiv$ product formed, and $a \equiv$ initial substrate, and $K, \Phi, m, \text{ and } m$ 
  were empirically derived constants. While under recognized during his time, Henri Victors 
  work would eventually be picked up 10 years later by Leonor Michaelis and Maud Menten.

  \subsection{Michaelis-Menten}
  While note the first to formulate a mathematical framework for enzymatic activity, Leonor
  Michaelis and Maud Menten were the first to formulate a detailed and bio-physically meaningful 
  model for enzyme activity. The following case describes the activity of an enzyme to
  produce one molecule count product from one molecule count substrate. This equation 
  recommended by the \textit{International Union of Biochemistry and Molecular Biology}, 
  IUBMB is:
  
\begin{equation}
  \nu = \frac{dp}{dt} = \frac{Va}{K_m+a}
\end{equation}

Where $\nu$ us the rate of product production per time $\frac{dp}{dt}$, $V$ (also written
as $V_{\max}$) is the maximum production rate (more accurately the limit of production as
substrate concentration reaches saturation) and $K_m$ is the Michaelis Constant. The
physical interpretation of the Michaelis-Constant is the concentration of substrate when 
$\nu={0.5}V$. This seemingly simplified description of enzymatic activity would go on to 
be widely used in descriptions biomolecular reactions.

The assumptions of the Michaealis-Menten model are as follows:

\end{document}

%-----------------------------------------------------------------------------------------
% This is meant to be a practice and practicle expercise in using latex to form chemically
% relevent figures. The paackage Chemfig will be used to form a basis which will then be 
% touched up in either photoshop or illlustrator. The subject is an dipeptide of RNA which
% showcases paradines and parimadines.
% Author: Alec Taylor 
% Date:
%-----------------------------------------------------------------------------------------
\documentclass{a0poster}
\usepackage{chemfig}
\begin{document}

\begin{figure}
  \definesubmol\phosphate{P(-[0]O)(-[-2]O)(=[-4]O)}
  \definesubmol\ribose{-O-[2,0.5,2]?<[7,0.7]-[,,,,,line width=5pt](-[-2]!\phosphate)-
  -([-2,0.7]OH)>[1,0.7]O-[3,0.7]?}
  \chemfig[cram width=5pt]{-O-[-1]-[-2]!\ribose}
\end{figure}

%\begin{figure}
%  \chemfig[cram width=5pt]{(-O-[-1]-*5(?<[line width =5pt]->-O?))-[0]-    O=[-2]P(-[-4]O)(-[0]O)(-[-2]O)-[0]O-[-1]-
%    (*5(?<-[,,,,line width=5pt]>(-*5(-N-*6(-N= (- (NH_2))-(NH)-(=0)-) =-N=--[,,,3]))-O-[1]?))
%    }
%\end{figure}
\end{document}